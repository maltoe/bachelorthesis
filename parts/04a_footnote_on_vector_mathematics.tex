Let $\vec{p}$ be the vector belonging to the intersection point and $\vec{x}$ the vector pointing at the center of the circle (see figure 3 in~\cite[p. 264]{kuruoglu2009aml}), while $\vec{xp} = \vec{p} - \vec{x}$ is the direction vector from Point $p$ to the center point $x$. We could then determine the intersection $a$ of the line segment between $x$ and $p$ by scaling this direction vector to the circle's radius which means multiplying the vector by $\frac{r}{\left| \vec{xp} \right|}$, i.e. as a vector: $\vec{a} = \vec{x} + \frac{r}{\left| \vec{xp}\right| } \cdot \vec{xp}$. The middle of the line segment $\overline{pa}$ can be calculated as $\vec{p'} = \vec{p} + \frac{1}{2} \cdot \vec{pa} = \vec{p} + \frac{1}{2} \cdot \left( \left( \vec{x} + \frac{r}{\left|\vec{xp}\right|} \cdot \vec{xp} \right) - \vec{p} \right)$.