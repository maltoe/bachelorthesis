\begin{abstract}
While clock rates of modern general purpose processors seem to have reached 
their peak at around 4 Gigahertz, exploitation of extensions to the instruction set 
such as SSE and AVX has become more and more important in performance critical applications. 
Most recent compilers make use of these instructions automatically when told to optimize an
application for speed. Still, it sometimes may be necessary to manually take advantage of 
CPU features that the compiler is not capable of using in the very situation. 
In the following bachelor thesis I am exploring ways to optimize performance of real-world
algorithms using techniques such as parallelization and manual cache organization. Based
on the example of a lateration algorithm simulator it will be shown how compiler intrinsics 
and code restructuring can be used to measurably improve execution performance.
\end{abstract}
\clearpage

